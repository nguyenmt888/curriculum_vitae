%-----------------------------------------------------------------------------------------------------------------------------------------------%
%	The MIT License (MIT)
%
%	Copyright (c) 2019 Jan Küster
%
%	Permission is hereby granted, free of charge, to any person obtaining a copy
%	of this software and associated documentation files (the "Software"), to deal
%	in the Software without restriction, including without limitation the rights
%	to use, copy, modify, merge, publish, distribute, sublicense, and/or sell
%	copies of the Software, and to permit persons to whom the Software is
%	furnished to do so, subject to the following conditions:
%	
%	THE SOFTWARE IS PROVIDED "AS IS", WITHOUT WARRANTY OF ANY KIND, EXPRESS OR
%	IMPLIED, INCLUDING BUT NOT LIMITED TO THE WARRANTIES OF MERCHANTABILITY,
%	FITNESS FOR A PARTICULAR PURPOSE AND NONINFRINGEMENT. IN NO EVENT SHALL THE
%	AUTHORS OR COPYRIGHT HOLDERS BE LIABLE FOR ANY CLAIM, DAMAGES OR OTHER
%	LIABILITY, WHETHER IN AN ACTION OF CONTRACT, TORT OR OTHERWISE, ARISING FROM,
%	OUT OF OR IN CONNECTION WITH THE SOFTWARE OR THE USE OR OTHER DEALINGS IN
%	THE SOFTWARE.
%	
%
%-----------------------------------------------------------------------------------------------------------------------------------------------%


%============================================================================%
%
%	DOCUMENT DEFINITION
%
%============================================================================%

%we use article class because we want to fully customize the page and don't use a cv template
\documentclass[10pt,A4]{article}	


%----------------------------------------------------------------------------------------
%	ENCODING
%----------------------------------------------------------------------------------------

% we use utf8 since we want to build from any machine
\usepackage[utf8]{inputenc}		

%----------------------------------------------------------------------------------------
%	LOGIC
%----------------------------------------------------------------------------------------

% provides \isempty test
\usepackage{xstring, xifthen}

%----------------------------------------------------------------------------------------
%	FONT BASICS
%----------------------------------------------------------------------------------------

% some tex-live fonts - choose your own

%\usepackage[defaultsans]{droidsans}
%\usepackage[default]{comfortaa}
%\usepackage{cmbright}
\usepackage[default]{raleway}
%\usepackage{fetamont}
%\usepackage[default]{gillius}
%\usepackage[light,math]{iwona}
%\usepackage[thin]{roboto} 

% set font default
\renewcommand*\familydefault{\sfdefault} 	
\usepackage[T1]{fontenc}

% more font size definitions
\usepackage{moresize}

% language
\usepackage[german]{babel}

% text alignment
\usepackage{ragged2e}

%----------------------------------------------------------------------------------------
%	FONT AWESOME ICONS
%---------------------------------------------------------------------------------------- 

% include the fontawesome icon set
\usepackage{fontawesome}

% use to vertically center content
% credits to: http://tex.stackexchange.com/questions/7219/how-to-vertically-center-two-images-next-to-each-other
\newcommand{\vcenteredinclude}[1]{\begingroup
\setbox0=\hbox{\includegraphics{#1}}%
\parbox{\wd0}{\box0}\endgroup}

% use to vertically center content
% credits to: http://tex.stackexchange.com/questions/7219/how-to-vertically-center-two-images-next-to-each-other
\newcommand*{\vcenteredhbox}[1]{\begingroup
\setbox0=\hbox{#1}\parbox{\wd0}{\box0}\endgroup}

% icon shortcut
\newcommand{\icon}[3] { 							
	\makebox(#2, #2){\textcolor{maincol}{\csname fa#1\endcsname}}
}	

% icon with text shortcut
\newcommand{\icontext}[4]{ 						
	\vcenteredhbox{\icon{#1}{#2}{#3}}  \hspace{2pt}  \parbox{0.9\mpwidth}{\textcolor{#4}{#3}}
}

% icon with website url
\newcommand{\iconhref}[5]{ 						
    \vcenteredhbox{\icon{#1}{#2}{#5}}  \hspace{2pt} \href{#4}{\textcolor{#5}{#3}}
}

% icon with email link
\newcommand{\iconemail}[5]{ 						
    \vcenteredhbox{\icon{#1}{#2}{#5}}  \hspace{2pt} \href{mailto:#4}{\textcolor{#5}{#3}}
}

%----------------------------------------------------------------------------------------
%	PAGE LAYOUT  DEFINITIONS
%----------------------------------------------------------------------------------------

% page outer frames (debug-only)
% \usepackage{showframe}		

% we use paracol to display breakable two columns
\usepackage{paracol}

% define page styles using geometry
\usepackage[a4paper]{geometry}

% remove all possible margins
\geometry{top=1cm, bottom=1cm, left=1cm, right=1cm}

\usepackage{fancyhdr}
\pagestyle{empty}

% space between header and content
% \setlength{\headheight}{0pt}

% indentation is zero
\setlength{\parindent}{0mm}

%----------------------------------------------------------------------------------------
%	TABLE /ARRAY DEFINITIONS
%---------------------------------------------------------------------------------------- 

% extended aligning of tabular cells
\usepackage{array}

% custom column right-align with fixed width
% use like p{size} but via x{size}
\newcolumntype{x}[1]{%
>{\raggedleft\hspace{0pt}}p{#1}}%


%----------------------------------------------------------------------------------------
%	GRAPHICS DEFINITIONS
%---------------------------------------------------------------------------------------- 

%for header image
\usepackage{graphicx}

% use this for floating figures
% \usepackage{wrapfig}
% \usepackage{float}
% \floatstyle{boxed} 
% \restylefloat{figure}

%for drawing graphics		
\usepackage{tikz}				
\usetikzlibrary{shapes, backgrounds,mindmap, trees}

\newsavebox{\picbox}
\newcommand{\cutpic}[3]{
  \savebox{\picbox}{\includegraphics[width=#2]{#3}}
  \tikz\node [draw, rounded corners=#1, line width=4pt,
    color=white, minimum width=\wd\picbox,
    minimum height=\ht\picbox, path picture={
      \node at (path picture bounding box.center) {
        \usebox{\picbox}};
    }] {};}

%----------------------------------------------------------------------------------------
%	Color DEFINITIONS
%---------------------------------------------------------------------------------------- 
\usepackage{transparent}
\usepackage{color}

% primary color (big background, date)
\definecolor{maincol}{RGB}{ 77, 75, 70 } 

% accent color, secondary
% \definecolor{accentcol}{RGB}{ 250, 150, 10 }

% dark color (big headline)
\definecolor{darkcol}{RGB}{ 179, 150, 35 }

% light color (vetical line and background of load beam)
\definecolor{lightcol}{RGB}{ 224, 224, 224 }


% Package for links, must be the last package used
\usepackage[hidelinks]{hyperref}

% returns minipage width minus two times \fboxsep
% to keep padding included in width calculations
% can also be used for other boxes / environments
\newcommand{\mpwidth}{\linewidth-\fboxsep-\fboxsep}
	


%============================================================================%
%
%	CV COMMANDS
%
%============================================================================%

%----------------------------------------------------------------------------------------
%	 CV LIST
%----------------------------------------------------------------------------------------

% renders a standard latex list but abstracts away the environment definition (begin/end)
\newcommand{\cvlist}[1] {
	\begin{itemize}{#1}\end{itemize}
}

%----------------------------------------------------------------------------------------
%	 CV TEXT
%----------------------------------------------------------------------------------------

% base class to wrap any text based stuff here. Renders like a paragraph.
% Allows complex commands to be passed, too.
% param 1: *any
\newcommand{\cvtext}[1] {
	\begin{tabular*}{1\mpwidth}{p{0.98\mpwidth}}
		\parbox{1\mpwidth}{#1}
	\end{tabular*}
}

%----------------------------------------------------------------------------------------
%	CV SECTION
%----------------------------------------------------------------------------------------

% Renders a a CV section headline with a nice underline in main color.
% param 1: section title
\newcommand{\cvsection}[1] {
	\vspace{14pt}
	\cvtext{
		\textbf{\LARGE{\textcolor{darkcol}{\uppercase{#1}}}}\\[-4pt]
		\textcolor{maincol}{ \rule{0.1\textwidth}{2pt} } \\
	}
}

%----------------------------------------------------------------------------------------
%	META SKILL
%----------------------------------------------------------------------------------------

% Renders a progress-bar to indicate a certain skill in percent.
% param 1: name of the skill / tech / etc.
% param 2: level (for example in years)
% param 3: percent, values range from 0 to 1
\newcommand{\cvskill}[3] {
	\begin{tabular*}{1\mpwidth}{p{0.72\mpwidth}  r}
 		\textcolor{black}{\textbf{#1}} & \textcolor{maincol}{#2}\\
	\end{tabular*}%
	
	\hspace{4pt}
	\begin{tikzpicture}[scale=1,rounded corners=2pt,very thin]
		\fill [lightcol] (0,0) rectangle (1\mpwidth, 0.15);
		\fill [maincol] (0,0) rectangle (#3\mpwidth, 0.15);
  	\end{tikzpicture}%
}


%----------------------------------------------------------------------------------------
%	 CV EVENT
%----------------------------------------------------------------------------------------

% Renders a table and a paragraph (cvtext) wrapped in a parbox (to ensure minimum content
% is glued together when a pagebreak appears).
% Additional Information can be passed in text or list form (or other environments).
% the work you did
% param 1: time-frame i.e. Sep 14 - Jan 15 etc.
% param 2:	 event name (job position etc.)
% param 3: Customer, Employer, Industry
% param 4: Short description
% param 5: work done (optional)
% param 6: technologies include (optional)
% param 7: achievements (optional)
\newcommand{\cvevent}[7] {
	
	% we wrap this part in a parbox, so title and description are not separated on a pagebreak
	% if you need more control on page breaks, remove the parbox
	\parbox{\mpwidth}{
		\begin{tabular*}{1\mpwidth}{p{0.72\mpwidth}  r}
	 		\textcolor{black}{\textbf{#2}} & \colorbox{maincol}{\makebox[0.25\mpwidth]{\textcolor{white}{#1}}} \\
			\textcolor{maincol}{\textbf{#3}} & \\
		\end{tabular*}\\[4pt]
	
		\ifthenelse{\isempty{#4}}{}{
			\cvtext{#4}\\
		}
	}

	\ifthenelse{\isempty{#5}}{}{
		\vspace{4pt}
		{#5}
	}
	\vspace{4pt}
}

%----------------------------------------------------------------------------------------
%	 CV META EVENT
%----------------------------------------------------------------------------------------

% Renders a CV event on the sidebar
% param 1: title
% param 2: subtitle (optional)
% param 3: customer, employer, etc,. (optional)
% param 4: info text (optional)
\newcommand{\cvmetaevent}[4] {
	\textcolor{maincol} {\cvtext{\textbf{\begin{flushleft}#1\end{flushleft}}}}

	\ifthenelse{\isempty{#2}}{}{
	\textcolor{darkcol} {\cvtext{\textbf{#2}} }
	}

	\ifthenelse{\isempty{#3}}{}{
		\cvtext{{ \textcolor{darkcol} {#3} }}\\
	}

	\cvtext{#4}\\[14pt]
}

%---------------------------------------------------------------------------------------
%	QR CODE
%----------------------------------------------------------------------------------------

% Renders a qrcode image (centered, relative to the parentwidth)
% param 1: percent width, from 0 to 1
\newcommand{\cvqrcode}[1] {
	\begin{center}
		\includegraphics[width={#1}\mpwidth]{qrcode}
	\end{center}
}

%=+=+=+=+=+=+=+=+=+=+=+=+=+=+=+=+=+=+=+=+=+=+=+=+=+=+=+=+=+=+=+=+=+=+=+=+=+=+=+=+
%,,,,,,,,,,,,,,,,,,,,,,,,,,,,,,,,,,,,,,,,,,,,,,,,,,,,,,,,,,,,,,,,,,,,,,,,,,,,,,,,
                       % EDIT AFTER THIS POINT
%''''''''''''''''''''''''''''''''''''''''''''''''''''''''''''''''''''''''''''''''
%=+=+=+=+=+=+=+=+=+=+=+=+=+=+=+=+=+=+=+=+=+=+=+=+=+=+=+=+=+=+=+=+=+=+=+=+=+=+=+=+


%============================================================================%
%
%
%
%	DOCUMENT CONTENT
%
%
%
%============================================================================%
\begin{document}
\columnratio{0.31}
\setlength{\columnsep}{2.2em}
\setlength{\columnseprule}{4pt}
\colseprulecolor{lightcol}
\begin{paracol}{2}
\begin{leftcolumn}
%---------------------------------------------------------------------------------------
%	META IMAGE
%----------------------------------------------------------------------------------------

\cutpic{0cm}{\linewidth}{assets/images/profile_picture.jpeg}
\vfill\null

\cvsection{Kontakt}

\icontext{Phone}{14}{+49 176 62509296}{black}\\[6pt]
\iconemail{EnvelopeSquare}{14}{minh-tri.nguyen@gmx.net}{minh-tri.nguyen@gmx.net}{black}\\[6pt]
\icontext{Xing}{14}{\href{https://www.xing.com/profile/MinhTri_Nguyen8/portfolio}{MinhTri$\_$Nguyen}}{black}\\[6pt]
\icontext{Linkedin}{14}{\href{https://www.linkedin.com/in/dr-minh-tri-nguyen}{dr-minh-tri-nguyen}}{black}\\[6pt]
\icontext{MapMarker}{14}{{Vaihinger Str. 94a \\ 70567 Stuttgart}}{black}\\[6pt]
\icontext{BirthdayCake}{14}{{10. Oktober 1986, Wertheim}}{black}\\[6pt]
\vfill\null
%\cvqrcode{0.7}

%---------------------------------------------------------------------------------------
%	META SKILLS
%----------------------------------------------------------------------------------------
\cvsection{Fähigkeiten}

%\cvskill{Skill_Name} {Years of experience} {percentage of bar fill} \\[-2pt]

\cvskill{Sparx EA, Rhapsody, SysML} {4+ Jahre} {0.7} \\[-2pt]

\cvskill{DOORS Next Generation, Sphinx} {3+ Jahre} {0.7} \\[-2pt]

\cvskill{Python, MATLAB Simulink} {5+ Jahre} {0.9} \\[-2pt]

\cvskill{IPG CarMaker, MKS} {3+ Jahre} {0.7} \\[-2pt]

\cvskill{Github, MS Azure} {2+ Jahr} {0.5} \\[-2pt]

\cvskill{Scrum Master, SAFe} {2+ Jahr} {0.5} \\[-2pt]

\cvskill{Sprachen: Dt., engl. } {} {} \\[-2pt]

\cvskill{Führerscheinkl.: A, B} {} {} \\[-2pt]
\vfill\null
%\cvqrcode{0.7}

%---------------------------------------------------------------------------------------
%	ACHIEVEMENTS
%----------------------------------------------------------------------------------------
\cvsection{Engagement}

\cvtext{FAT Arbeitskreis AK20 Fahrdynamik}\\[12pt]
\cvtext{DOSB C-Trainer Leistungssport Boxen}\\[12pt]
\cvtext{DOSB C-Trainer Kraft und Fitness}\\[12pt]
\cvtext{Schöffe Amtsgericht Stuttgart}\\[12pt]
\vfill\null

\end{leftcolumn}
\begin{rightcolumn}

%---------------------------------------------------------------------------------------
%	TITLE  HEADER
%----------------------------------------------------------------------------------------
\fcolorbox{white}{maincol}{\begin{minipage}[c][3.5cm][c]{1\mpwidth}
	\begin {center}
		\HUGE{ \textbf{ \textcolor{white}{ \uppercase{ Minh-Tri Nguyen } } } } \\[-24pt]
		\textcolor{white}{ \rule{0.1\textwidth}{1.25pt} } \\[4pt]
		\large{ \textcolor{white} {System Engineer } }
	\end {center}
\end{minipage}} \\[14pt]
\vspace{-12pt}

%---------------------------------------------------------------------------------------
%	ABOUT ME
%----------------------------------------------------------------------------------------
\vfill\null
\cvsection{Über mich}

\cvtext{Die Art und Weise wie sich das zukünftige Automobil bewegt, die Gefühle, die es uns dabei vermittelt und die Rolle, die wir beim Erfüllen der Fahraufgabe einnehmen, bilden die Grundlage meines Antriebs.}\\[6pt]
\cvtext{Mit der Erfahrung in der Entwicklung hochautomatisierter Fahrsysteme sowie der Expertise im Bereich der Fahrdynamik und des Fahrkomforts glaube ich fest daran, Sie bei Ihren Herausforderungen tatkräftig unterstützen zu können.}
\vfill\null

%---------------------------------------------------------------------------------------
%	WORK EXPERIENCE
%----------------------------------------------------------------------------------------
\vfill\null    
\cvsection{Berufliche Erfahrung}

\cvevent
	{\textbf{02/19 - heute}}
	{Systemingenieur / Systemarchitekt - Entwicklung autonomer Fahrsysteme SAE L4}
	{Robert Bosch GmbH}
	{\begin{itemize}
			\item Spezifikation des Fahrzeugverhaltens unter Berücksichtigung der funktionalen Sicherheit nach ISO 26262 und Definition der Anforderungen nach IREB
			\item Erstellung der funktionalen und logischen Architektur gemäß MBSE und Ableitung der SW-Anforderungen
			\item Spezifikation logischer Testszenarien und Auswertung von Simulations- und Testergebnissen
			\item Berechnung der notwendigen Sensorreichweiten zur Definition des Sensorsets
			\item Planung und Durchführung agiler Arbeitsmethoden nach Scrum und SAFe 
	\end{itemize}}
\vfill\null

\cvevent
	{\textbf{02/13 – 01/19}}
	{Wissenschaftlicher Mitarbeiter - Fahrzeugtechnik und Fahrdynamik}
	{IVK - Universität Stuttgart}
	{\begin{itemize}
		\item Messtechnische Ausrüstung der Prototypenfahrzeuge sowie die Planung und Durchführung der Fahrversuche
		\item Methodenentwicklung für subjektive Fahrdynamik- und Komfortbewertungen mit dem digitalen Prototyp im Stuttgarter Fahrsimulator des FKFS
		\item Modellierungen der Fahrzeugdynamik und des vestibulären Wahrnehmungssystems sowie die Funktionsentwicklung verschiedener Fahrerassistenzsysteme
		\item Lehrtätigkeit am Institut IVK im Bereich der Fahrzeugtechnik und Fahrdynamik
	\end{itemize}}
\vfill\null

%---------------------------------------------------------------------------------------
%	EDUCATION
%----------------------------------------------------------------------------------------
%\vfill\null
\cvsection{Bildungsweg}

\cvevent
	{\textbf{02/13 - 07/19}}
	{Promotion, Note: \textbf{magna cum laude}}
	{IVK - Universität Stuttgart}
	{Dissertation: \href{https://www.springerprofessional.de/subjektive-wahrnehmung-und-bewertung-fahrbahninduzierter-gier-un/17938230}{Subjektive Wahrnehmung und Bewertung fahrbahninduzierter Gier- und Wankbewegungen im virtuellen Fahrversuch}}
\vfill\null

\cvevent
	{\textbf{10/09 - 11/12}}
	{Diplom - Fahrzeug- und Motorentechnik, Note: 2,2}
	{TU Münchnen}
	{Diplomarbeit - BMW AG: Einflussanalyse ausgewählter Fahrwerksparameter auf Übergangsfahreigenschaften und Aufbauschwingungskomfort, Note: 1,3}
\vfill\null

\cvevent
	{\textbf{10/06 - 09/09}}
	{Vordiplom - Fahrzeug- und Motorentechnik}
	{Universität Stuttgart}
	{Praktikum - Dr. Ing. h.c. F. Porsche AG: Entwicklung Gesamtfahrwerk}
\vfill\null

\cvevent
	{\textbf{06/06}}
	{Abitur, Note: 2,1}
	{Technisches Gymnasium Wertheim}
	{}
\vfill\null


%---------------------------------------------------------------------------------------
%	PUBLICATION
%----------------------------------------------------------------------------------------
\vfill\null
\cvsection{Veröffentlichungen}

\cvevent
	{\textbf{06/17}}
	{\href{https://link.springer.com/chapter/10.1007/978-3-658-18459-9_19}{Subjective testing of a torque vectoring approach based on driving characteristics in the driving simulator}}
	{J. ATZ live chassis.tech plus, München, Juni 2017}
	{}
\vfill\null

\cvevent
	{\textbf{03/17}}
	{\href{https://www.sae.org/publications/technical-papers/content/2017-01-1564/}{Subjective perception and evaluation of driving dynamics in the virtual test drive}}
	{J. SAE Int. J. Veh. Dyn., Stab., and NVH, Detroit, März 2017}
	{}
\vfill\null

\cvevent
	{\textbf{02/16}}
	{\href{https://link.springer.com/article/10.1007/s38311-014-0024-3}{Simulation of driving under unsteady crosswind conditions}}
	{ATZ Springer Vieweg, Februar 2016}
	{}
\vfill\null

\cvevent
	{\textbf{09/14}}
	{\href{http://dsc2015.tuebingen.mpg.de/Docs/DSC_Proceedings/2014/DSC14_31_Pitz.pdf}{Combined motion of a hexapod with xy-table system for lateral movements}}
	{Driving Simulator Conference, Paris, September 2014}
	{}
\vfill\null

%\cvevent
%	{\textbf{20XX}}
%	{Project Name}
%	{Tool: Web Development}
%	{A short description of your project.}
%\vfill\null
%---------------------------------------------------------------------------------------
%	PERSONAL DETAILS
%----------------------------------------------------------------------------------------
%\vfill\null
%\cvsection{EXTRACURRICULAR}
%\vspace{-0.3cm}
%\begin{itemize}
%  \item Als Ausgleich zu meiner logischen Arbeit mache ich \textbf{Musik}.
%  \item Ein weiteres Hobby von mir ist \textbf{Videos oder Fotos bearbeiten}.
%\end{itemize}
%\vfill\null


% hotfixes to create fake-space to ensure the whole height is used
\vfill
\vfill
\vfill
\end{rightcolumn}
\end{paracol}
\end{document}

